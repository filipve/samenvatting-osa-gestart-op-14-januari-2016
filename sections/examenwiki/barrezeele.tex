\section{Barrezeele}

\subsection{Mondeling}

\subsubsection{beschrijving van een bedrijf/organisatie (nmbs)}

\be
\itf Maak een klassendiagram
\itf Maak een toestandsdiagram
\itf Maak een use-case beschrijving
\itf Maak een sequentiediagram
\ee

\subsubsection{Gegeven: een case study van een taxibedrijf.}

\be
\itf het klassendiagram
\itf het toestandsdiagram voor Klant
\itf de use case beschrijving voor 'plaats reservatie'
\itf het sequentiediagram voor 'plaats reservatie'
\ee 

\subsubsection{Je moet een systeem voor de Olympische Spelen maken}

\be
\itf Maak een klassediagramma
\itf Maak een toestandsdiagram voor 'Deelname'
\itf Maak een use-case beschrijving voor 'Voer prestaties in na wedstrijd'
\itf Maak een sequentiediagram voor 'Voer prestaties in na wedstrijd'
\itf 
\ee

\subsubsection{Je krijgt een hele beschrijving van een racecircuit.}

\be
\itf Maak een klassediagramma in de domein(dinges)fase
\itf Maak een Use Case beschrijving voor INSCHRIJVEN VOOR RACE
\itf Maak een toestandsdiagramma voor INSCHRIJVEN
\itf Maak een sequentiediagramma voor INSCHRIJVEN
\itf 
\ee

\subsection{Schriftelijk}

%Question environment: \begin{question}[<options>]{<points>}
\begin{question}
Welk zijn de implementatiediagrammen, waarvoor worden ze gebruikt, wat is hun nut, geef een voorbeeld
\end{question}

\begin{solution}[print]

\end{solution}
%Question environment: \begin{question}[<options>]{<points>}
\begin{question}
Welke diagrammen behoren tot de implementatiediagrammen? Welke info bevatten deze diagrammen? Wat is het nut hiervan?
\end{question}

\begin{solution}[print]

\end{solution}

%Question environment: \begin{question}[<options>]{<points>}
\begin{question}
Geef de implementatiediagrammen, wanneer stel je deze op, wat geven deze weer, wat zijn de verschillen hiertussen, geef een voorbeeld met de juiste symbolen.
\end{question}

\begin{solution}[print]

\end{solution}

%Question environment: \begin{question}[<options>]{<points>}
\begin{question}
Wat is aan activiteitendiagram, waarvoor dient het, waarvoor wordt het gebruikt, geef een voorbeeld
\end{question}

\begin{solution}[print]

\end{solution}

%Question environment: \begin{question}[<options>]{<points>}
\begin{question}
Wat is een activiteits diagram, geef een voorbeeld aan de hand van de probleemstelling van het mondeling.
\end{question}

\begin{solution}[print]

\end{solution}

%Question environment: \begin{question}[<options>]{<points>}
\begin{question}
Wat is het verschil tussen statische en dynamische diagrammen? Geef voor elk gezien diagram aan of het statisch of dynamisch is, en verklaar.
\end{question}

\begin{solution}[print]

\end{solution}

%Question environment: \begin{question}[<options>]{<points>}
\begin{question}
Wat is het componentdiagram en waar kan het voor gebruikt worden? Wat is in UML de guard en in welke diagrammen kan je het tegenkomen?
\end{question}

\begin{solution}[print]

\end{solution}

%Question environment: \begin{question}[<options>]{<points>}
\begin{question}
Wat betekent aggregatie? Wat betekent compositie? Wat zijn de verschillen en gelijkenissen? Illustreer aan de hand van een goed gekozen voorbeeld.
\end{question}

\begin{solution}[print]

\end{solution}

%Question environment: \begin{question}[<options>]{<points>}
\begin{question}
 Waartoe dient een use-case diagram? Wat is het nut hiervan? Heeft dit diagram invloed op andere diagrammen? Zo ja, zeg specifiek op welke, en op welke manier.
\end{question}

\begin{solution}[print]

\end{solution}

%Question environment: \begin{question}[<options>]{<points>}
\begin{question}

\end{question}

\begin{solution}[print]

\end{solution}

%Question environment: \begin{question}[<options>]{<points>}
\begin{question}
Wat is een actor? In welke UML's komt deze voor? Hoe wordt deze weergegeven
\end{question}

\begin{solution}[print]

\end{solution}

%Question environment: \begin{question}[<options>]{<points>}
\begin{question}
Wat is meervoudige overerving en herhaaldelijke overerving? Geef een goed voorbeeld.
\end{question}

\begin{solution}[print]

\end{solution}

%Question environment: \begin{question}[<options>]{<points>}
\begin{question}
Bij de use-cases hebben een black-box van het systeem wanneer wordt deze een white-box.
\end{question}

\begin{solution}[print]

\end{solution}

%Question environment: \begin{question}[<options>]{<points>}
\begin{question}
UML is een iteratieve methode, bij welke diagrammen gebeurt dit.
\end{question}

\begin{solution}[print]

\end{solution}

\subsubsection{Terminologie}


