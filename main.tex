\documentclass[a4paper,11pt]{article}
\usepackage{import}
%\usepackage{projectstyle}

%de naam hier is de naam van de file in de sty folder niet de naam van providespackage
\usepackage{sty/1loadrequiredpackagesatveryfirst}
\usepackage{sty/2metadata}
\usepackage{sty/3tocstyle}
\usepackage{sty/4appendixstyle}
\usepackage{sty/5colorstyles}
\usepackage{sty/6sectionsstyle}
\usepackage{sty/7newcommandsfilip}
\usepackage{sty/8glossariesstyle}
\usepackage{sty/9imagesstyle}
\usepackage{sty/10liststyles}
\usepackage{sty/11headerfooterstyle}
\usepackage{sty/12mathstyles}
\usepackage{sty/13customstylesfromothersites}
\usepackage{sty/14stylesforalgoritmsandsourcecode}
\usepackage{sty/15stylefortodonotes}
%\usepackage{sty/}
%\usepackage{sty/}
%\usepackage{sty/}
%\usepackage{sty/}
%\usepackage{sty/}


\usepackage{makeidx}


%It prints out all index entries in the left margin of the text. This is quite useful for proofreading a document and verifying the index. For more information, see the Indexing section.

%https://en.wikibooks.org/wiki/LaTeX/Indexing

\usepackage{showidx} %It prints out all index entries in the left margin of the text.

\makeindex

%we have to define a bibliography style in the preamble
\bibliographystyle{plain}

\usepackage{booktabs}

\usepackage{exsheets}

\DeclareTranslation{Dutch}{exsheets-exercise-name}{Vraag}
\DeclareTranslation{Dutch}{exsheets-question-name}{opgave}
\DeclareTranslation{Dutch}{exsheets-solution-name}{Oplossing}

\usepackage{caption}

\begin{document}


%\glsaddall
%\import{./}{title.tex}

%\clearpage

\import{./}{title2.tex}

\clearpage

\thispagestyle{empty}

\tableofcontents

\clearpage

\begin{comment}

%hoofdstuk 4
%\setcounter{section}{6}
%\newpage
%\section{Multimedia netwerken}
\import{sections/lessen/}{1systemsanalysis.tex}

\end{comment}


\begin{comment}

\import{sections/lessen/}{1informatiesystemen.tex}

\import{sections/lessen/}{2uml.tex}

\import{sections/lessen/}{3buziness.tex}

\import{sections/lessen/}{4domainmodel.tex}

\import{sections/lessen/}{5requirementsmodel.tex}



\end{comment}

\import{sections/lessen/}{6dynamischmodelleren.tex}

\begin{comment}


\import{sections/lessen/}{7applicatielaag.tex}
\import{sections/lessen/}{8implementationdiagram.tex}
\import{sections/lessen/}{8deploymentdiagram.tex}

\end{comment}

\section{examenwiki vragen}
\import{sections/examenwiki/}{barrezeele.tex}
\import{sections/examenwiki/}{fox.tex}
\import{sections/examenwiki/}{martens.tex}



%Next we are adding the additional lists to the end of the document:

\listofalgorithms
\clearpage
\listoffigures
\clearpage
\listoftables
\clearpage

%\printglossaries
\printglossary[title=Termen,toctitle=Lijst van termen]

\printglossary[type=\acronymtype]


\clearpage
%\import{./}{bibliography.tex}
\printsolutions

\end{document}
